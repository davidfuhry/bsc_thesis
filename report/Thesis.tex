% !TeX program = xelatex
% !BIB program = biber

\documentclass{article}

\usepackage{longtable}
\usepackage{array}
\usepackage{placeins}
\usepackage{csquotes}
\usepackage{hyperref}
\usepackage{graphicx}
\usepackage{url}
\usepackage{todonotes}
\usepackage{booktabs}
\usepackage{tablefootnote}
\usepackage{siunitx}
\usepackage{multirow}
\usepackage{tabularx}
\sisetup{output-exponent-marker=\ensuremath{\mathrm{e}}}
\usepackage[english, german, ngerman]{babel}
\usepackage[
    backend=biber,
    style=apa,
    autocite=inline,
    mincitenames=1,
    maxcitenames=2
]{biblatex}

\addbibresource{thesis-literature.bib}
\addbibresource{thesis-software.bib}
\addbibresource{thesis-online.bib}



\selectlanguage{ngerman}

\def\UrlFont{\rmfamily}

\begin{document}

\begin{titlepage}
    \begin{center}
        \vspace*{1cm}

        \textbf{\Large Automatisierte Erkennung von verschwörungstheoretischen Artikeln}

        \vspace{0.5cm}
        Mit Methoden der natürlichen Sprachverarbeitung und des maschinellen Lernens
                
        \vspace{1.5cm}


        \vfill
                
        Bachelor-Arbeit zur Erlangung des akademischen Grades eines\\
        B.Sc. Digital Humanities

        \vspace{3cm}
                
        Eingereicht von: David Fuhry
        
        \vspace{0.2cm}

        Matrikel-Nr. 3704472\\
        Zschochersche Str. 32\\
        04229 Leipzig

        \vspace{1cm}

        Betreuer: Jun.-Prof. Dr. Phil. Manuel Burghardt
        
        \vspace{0.2cm}
                    
        Computational Humanities\\
        Institut für Informatik\\
        Universität Leipzig\\
        Augustusplatz 10\\
        04109 Leipzig 
                
    \end{center}
\end{titlepage}

\tableofcontents

\thispagestyle{empty}

\newpage

% Introduction

\pagenumbering{arabic}

\section{Einleitung}

Verschwörungstheorien sind aktuell stark im Zentrum der öffentlichen Aufmerksamkeit.
Während dieser Aufmerksamkeitsschub vor allem auf die Covid-19 Pandemie zurückzuführen ist, erleben Verschwörungstheorien schon seit Anfang der 2000er Jahre einen kontinuierlichen Popularitätsanstieg.
Dieser ist vor allem begründet in der zunehmenden Verbreitung des Internets und der sozialen Medien \parencite[vgl.][492]{stano_2020}.

Eine der Kernfragen in der wissenschaftlichen Auseinandersetzung mit Verschwörungstheorien ist die Frage was eine Verschwörungstheorie ist und was nicht.
Eine theoretische Auseinandersetzung mit dieser Frage findet insbesondere in der Wissensoziologie und der Philosophie statt.\footnote{Siehe z.B. \textcite[]{coady_2006} oder auch \textcite[23-53]{uscinski_2014}.}

Weitgehend unbeantwortet ist hingegen, wie sich diese Frage quantitativ beantworten lässt.
Die wenigen hierzu veröffentlichten Arbeiten nutzen meist verhältnismäßig spezifische und komplexe Methoden.\footnote{Siehe etwa \textcite[]{samory_2018} oder \textcite[]{shahsavari_2020}.}

Diese Arbeit stellt einen Ansatz vor, der versucht dieses Problem mit Methoden des Text-Minings und des maschinellen Lernens zu lösen.
Der vorgestellte Ansatz ist dabei deutlich allgemeiner als die in der Literatur vertretenen und identifiziert verschwörungstheoretische Artikel mittels sprachlicher und stilistischer Merkmale.

Es wird dafür ein Korpus aus Artikeln von sieben Webportalen der deutschsprachigen, verschwörungstheoretischen Szene ausgewertet. 
Aus diesem werden Features extrahiert, die anschließend genutzt werden, um ein Modell zu trainieren, welches diese Artikel von (wissenschafts-) journalistischen Artikeln in einem Vergleichskorpus unterscheiden kann.
Dabei erfolgt ein Rückgriff auf in der bestehenden, überwiegend qualitativen Forschung gemachten Erkenntnisse und Beobachtungen.
Ebenfalls werden Methoden aus der natürlichen Sprachverarbeitung verwendet, um Features aus Wortfrequenzen und Part-of-Speech Tags zu erstellen.

Das erstellte Modell kann präzise verschwörungstheoretische Artikel identifizieren und von denen im Vergleichskorpus unterscheiden.
Die dabei gemachten Beobachtungen können dabei helfen in der qualitativen Forschung gemachte Beobachtungen quantitativ zu validieren.
Auch könnte die hier vorgestellte Methode hilfreich sein um verschwörungstheoretische Inhalte in größeren Datenmengen zu identifizieren.

% Theorie

\section{Theorie}

% Forschungsstand, Methoden zur Identifizierung

\subsection{Stand der Forschung}

Der Beginn der Wissenschaftlichen auseinandersetzung mit Verschwörungstheorien wird üblicherweise rund um das Erscheinen von Richard \textcite{hofstadter_2008} Essay \textit{The Paranoid Style in American Politics} verortet.
In diesem setzt Hofstadter sich mit dem was er für von Verschwörungstheorien beeinflusste Politik in den USA hält, wie etwa die Schriften von Senator McCarthy und die nach ihm benannte Praxis des McCarthyismus \todo{Maybe citation needed}.

Das weitere Forschungsinteresse im restlichen 20. Jahrhundert hielt sich in Grenzen, die Verbleibende Forschung fiel meist in eine von Zwei Kategorien, wie \textcite{sunstein_2008} passend zusammenfassen:

\begin{quotation}
    The academic literature on conspiracy theories is thin, and most of it falls into one 
    of two classes: (1) work by analytic philosophers, especially in epistemology and the 
    philosophy of science, that asks what counts as a “conspiracy theory” and whether such theories are methodologically suspect; (2) a smattering of work in sociology and Freudian psychology on the causes of conspiracy theorizing. \parencite[][2]{sunstein_2008}
\end{quotation}

In der jüngeren Vergangenheit haben vor allem Drei Entwicklungen Verschwörungstheorien oder zumindest deren Sichtbarkeit und damit auch der Forschung in diesem Bereich vorschub geleistet.

Dies wäre zum einen die Verbreitung des Internets um die Jahrtausendwende.
Die Mehrheit der Autoren sieht hier darin ein erhebliches Hilfsmittel für die Verbreitung von Verschwörungstheorien, wie etwa \textcite{stano_2020} schreibt: "[...] The Internet, and in particular social networks, have proved fundamental to the spread and development of such [conspiracy] theories" \parencite[][492]{stano_2020}.\footnote{Es gilt jedoch anzumerken, dass auch die gegenteile Auffassung in der Literatur vertreten ist, so etwa bei \textcite{clarke_2007}.}

In der jüngsten Vergangenheit haben besonders die Wahl Donald Trumps zum US-Präsidenten 2016 und die aktuelle COVID-19 Pandemie Verschwörungstheorien ins öffentliche Interesse gerückt.
Wärend die wissenschaftliche Aufarbeitung insbesondere von letzterer noch am Anfang steht, sind hier bereits erste Arbeiten vorhanden die neue Ansätze zeigen, wie etwa die von \textcite{shahsavari_2020}.

Eines der Kernprobleme der Forschungs zu Verschwörungstheorien, ist die Frage wie Verschwörungstheorien als solche zu identifizieren sind.
Diese Frage hat nicht nur eine fundamentale Rolle für theoretische Arbeiten zu Verschwörungstheorien, sondern ist auch von immenser praktischer Relevanz in einer Zeit in der in sozialen Medien die Grenzen zwischen Nachrichten, Verschwörungstheorien und Fake-News immer schwerer auszumachen sind.\todo{Eventuell google algorithmus zitieren}

In der Vergangenheit war die wissenschaftliche Beantwortung dieser Frage meist eng mit der Frage danach verbunden was eine Verschwörungstheorie ausmacht und anhand welcher kriterien sich dies bestimmen lässt.
Eine Beispielhafte Arbeit ist hier \textcite{uscinski_2014}.
Die Autoren nutzen darin 6 verschiedene Tests um Verschwörungstheorien zu identifizieren, als Beispiel sei Occamm`s Razor, also die Frage nach der einfachsten Erklärung, und die falsifizierbarkeit einer Theorie genannt.
Die Autoren selbst werten mit diesen Krieterien und der Methode der Inhaltsanalyse sowie einer Vielzahl von studentischen Kodierer:Innen einen Korpus an Briefen an US-Redaktionen aus \parencite[54ff]{uscinski_2014}.
Während die Autoren überzeugend dafür argumentieren, alle der aufgestellten Kriterien für eine Klassifizerung einzusetzten \parencite[52f]{uscinski_2014}, ist keines der eingeführten Kriterien, zumindest mit den aktuellen technischen möglichkeiten, wirklich geeignet um automatisierte Klassifikationen vorzunehmen.

Demgegenüber sind in jüngster Vergangenheit erste Arbeiten entstanden, die sich mit automatisieren Verfahren zum Erkennen von Verschwörungstheorien beschäftigen.
Hier ist vor allem die Arbeit von \textcite{samory_2018} hervorzuheben, in der die Autor:Innen automatisiert Triplets aus \textit{agent}, \textit{action} und \textit{target} aus Verschwörungstheoretischen Online Beiträgen extrahieren.
Als Datengrundlage nutzen sie dazu Beiträge aus dem Reddit Subforum r/conspiracy.
Diese Werten sie mittels einer NLP Pipeline aus die unter anderem Topic Modeling, Dependency Parser und Wordvektoren nutzt, aber auch vereinzelt Expertenwissen einbringt \parencite[][6ff]{samory_2018}.
Der Ansatz stellt einen wichtigen Beitrag zu einer vollständig automatisierten Analyse dar, ist aber auch verhältnismäßig komplex und basiert auf einer Datenbasis die nicht notwendigerweise Repräsentativ für "wild" vertretene Verschwörungstheorien ist, so dass die Übertragbarkeit für einen Identifizierungsmechanismus fragwürdig scheint.

Eine weitere Arbeit die sich mit automatisierten Auswertungen beschäftigt, ist die von \textcite{shahsavari_2020}.
Die Autor:innen werten dabei Online Diskussionen in Foren wie Reddit und 4Chan aus, die sich mit der Covid 19 Pandemie beschäftigen.
Die Auswahl ob Diskussionen verschwörungstheoretische Inhalte enthalten oder nicht, wurde dabei von einem Expertengremium getroffen \parencite[284f]{shahsavari_2020}.
Es werden mittels Methoden aus der Natürlichen Sprachverarbeitung narrative Netzwerke ausgewertet und so Verbindungen zu den inhalten von Nachrichten und den aktuell "beliebtesten" Verschwörungstheorien aufgezeigt. \todo{Hier eventuell mehr/besser belegen/recherchieren}
Während auch diese Arbeit zweifelsfrei einen wichtigen Beitrag zu einem besseren Verständnis insbesondere der Verbreitung von Verschwörungstheorien in sozialen Netzwerken darstellt, wird das Problem der identifizierung von Verschwörungstheorien damit nicht wirklich gelöst.

\subsection{Merkmale von verschwörungstheoretischen Texten}

Um ebendiese Aufgabe angehen zu können, soll zunächst ein Blick in die Literatur erfolgen und in die in der qualitativen und quantitativen Forschung gemachten Beobachtungen zu sprachlichen Merkmalen von Verschwörungstheorien.
Eine erschöpfende Auflistung aller in der Literatur zu findenden sprachlichen Merkmale von Verschwörungstheorien ist insbesondere mit dem erstarkten Forschungsinteresse in der jüngsten Vergangenheit kaum möglich noch wäre es zielführend.
Es sollen daher hier vor allem solche Merkmale genannt werden, die sich für eine quantitative Erfassung eignen und in dieser Arbeit 

Eine in der Literatur häufig gemachte Beobachtung ist die emotionalität der Argumentation in Verschwörungstheorien \parencite[Vgl.][10]{miller_2002}.
Dieses Argument findet sich auch bei \textcite[][93ff]{butter_2018}, der hier noch spezifischer darauf eingeht, dass es vor allem die vermeintlichen Verschwörer sind denen mit "metaphorisch aufgeladener, bisweilen apokalyptischer Sprache ausschließlich negative Eigenschaften zugeschrieben [werden]" \parencite[][93f]{butter_2018}.

Gleichzeitig spricht der Autor aber auch davon, dass sich Verschwörungstheoretiker traditionell um eine seriöse Darstellung bemühen und sich der verwendete Stil an den der Wissenschaft anlehne \parencite[][61]{butter_2018}.
Diese Feststellung findet sich bereits bei \textcite{hofstadter_2008}:

\begin{quotation}
    The higher paranoid scholarship is nothing if not coherent—in fact the paranoid mind is far more coherent than the real world. It is nothing if not scholarly in technique. McCarthy’s 96-page pamphlet, McCarthyism, contains no less than 313 footnote references, and Mr. Welch’s incredible assault on Eisenhower, The Politician, has one hundred pages of bibliography and notes. \parencite[][37]{hofstadter_2008}
\end{quotation}

Seit der Erstveröffentlichung von Hofstadters Aufsatz hat vor allem die Existenz des Internets zu Veränderungen in der Erzählart von Verschwörungstheorien geführt.
Die obsession an Belegen und Referenzen existiert zwar weiterhin, hat sich aber insoweit Verändert, als das nicht mehr unbedingt Fussnoten das Mittel der Wahl sind, vielmehr werden Inhalte direkt eingebunden oder verlinkt.
So schreibt etwa \textcite{soukup_2008} der sich spezifisch mit Verschwörungstheorien um den 11. September im Internet befasst, diese müssten als "digital, hypertextual, and multimedial experience" \parencite[10]{soukup_2008} verstanden werden.

Erst in der jüngeren Vergangenheit finden sich Arbeiten die sich spezifisch mit den linguistischen Merkmalen von Verschwörungstheorien auseinandersetzen.
\textcite{schafer_2018} etwa analysiert einen Korpus aus Kommentaren zu verschwörungstheoretischen YouTube-Videos.
Sie stellt dabei unter anderem eine gehäufte Verwendung von Ironie und \textit{scare quotes} fest, wenn es darum geht die Gegenseite negativ darzustellen \parencite[235]{schafer_2018}.
Ähnlich der bereits erörterten Literatur findet die Autorin auch eine hohe Dichte von Belegen und Referenzen die die Kompetenzen der VerschwörungstheoretikerInnen unterstreichen sollen, stellt aber noch heraus, dass diese sich häufig durch nennen von Namen und Zahlenangaben ausdrücken \parencite[234]{schafer_2018}.

Eine ähnliche Beobachtung macht auch \textcite{filatkina_2018}, welche feststellt:

\begin{quotation}
    In einem Beitrag oft mehrfach angeführte Verweise auf offizielle Zahlen [...] bzw. das wörtliche Zitieren der VertreterInnen der Wissenschaft und Politik sollen die Glaubwürdigkeit herstellen. \parencite[][208]{filatkina_2018}
\end{quotation}

Ebenfalls stellt die Autorin fest, dass eine übliche Konstruktion darin besteht, durch offene und geschlossene Fragen die Aufmerksamkeit der LeserInnen auf bestimmte Aspekte zu lenken \parencite[][205]{filatkina_2018}.

Schlussendlich findet sich bei \textcite[149]{stumpf_2019} noch die Feststellung, dass häufig Negationswörter verwendet werden um die offiziellen Aussagen zu verneinen.


% Vokabular:
%  - Entlarfungsvokabular \parencite[][51]{ebling_2013}
%  - Stumpf/Römer 2019 haben viel Vokabular

\subsection{Textklassifizierung}

% Hier über klassische Ansätze für Textklassifizierung etc.

% Vorgehen

\section{Vorgehen}

Als Grundlage dieser Arbeit dient ein Korpus aus Texten von deutschsprachigen Internetportalen, die der verschwörungstheoretischen, bzw. Truther Szene zuzuordnen sind.
Diese wurde im Verlauf der letzten Jahre über Webcrawling zusammengestellt.

Um einen strukturierte Analyse mit Methoden des maschinellen Lernens zu ermöglichen wird ein Vergleichskorpus mit Artikeln die dem Untersuchungskorpus von Form und Thematik möglich ähnlich gelagert sind, ohne aber dessen verschwörungstheoretischen Aspekte zu teilen.

Anschließend werden die Korpora gereinigt und mit Verfahren aus der natürlichen Sprachverarbeitung Features extrahiert auf denen Klassifizierungsverfahren arbeiten können.
Zum Einsatz kommen dabei sowohl Features die klassischerweise für Textklassifikation herangezogen werden wie Wortfrequenzen, Features die typischerweise eher für stilometrische Analysen verwendet werden wie Part-of-Speech Tags, sowie auf der vorhandenen Literatur zu Verschwörungstheorien basierende Features.
Die Implementierung aller Funktionen dieser Arbeit erfolgt in der Programmiersprache \textit{R} \parencite{r_2021}

Mit den so erstellten Daten soll abschließend ein Modell trainiert werden, dass zwischen den verschwörungstheoretischen Texten und dem Vergleichskorpus unterscheiden kann.
Um die Effektivität der auf der Literatur basierten Features einschätzen zu können wird ein Datensatz erstellt der nur diese Features enthält, sowie ein weiterer Datensatz der alle Features enthält.
Für jeden dieser Datensätze wird zunächst ein einfacher Entscheidungsbaum als Baseline Model trainiert.
Für das endgültige Modell kommt die \textit{LightGBM} Bibliothek \parencite[][]{lightgbm}, eine Implementierung von Gradient Boosting Descision Trees \parencite*{friedman_2002}.

Dieses Verfahren bietet neben der typischerweise relativ guten Präzision den Vorteil, dass die Ergebnisse, ähnlich einfachen Entscheidungsbäumen, verhältnismäßig gut interpretierbar sind.
Dies soll es Erlauben den Einfluss einzelner Features auf die Vorhersagegenauigkeit zu bestimmen, insbesondere auch der aus der Literatur abgeleiteten Features.

% Umsetzung

\section{Umsetzung}

\subsection{Datenbasis}

Als Grundlage dieser Arbeit wurde ein Korpus aus Texten von insgesamt 7 deutschsprachigen, der verschwörungstheoretiker Szene zuzurechnenden Internetangeboten erstellt.
\todo{R zitieren so nicht schon geschehen}
Die Skripte zum automatischen Abruf der Texte nutzen dafür das R-Paket \textit{rvest} \parencite{rvest} erstellt und extrahieren mittels für jede Seite seperat erstellten CSS-Selektoren und XPATH-Querys den jeweiligen Artikeltext. 
In diesem Vorgang wurden noch kleinere Bereinigungen an den Texten vorgenommen um wiederkehrende, nicht zum Artikeltext gehörende Elemente wie Werbung oder Spendenaufrufe zu entfernen, die Texte selbst wurden jedoch soweit möglich nicht weiter bearbeitet.
Weiterhin erfasst wurde zu den Artikeln das (angegebene) Veröffentlichungsdatum, der Artikeltitel sowie die Rubrik so diese Angabe vorhanden waren.

\begin{figure}[h]
    \centering
    \includegraphics[scale=0.45]{graphics/cons_freq_time.jpg}
    \caption{Anzahl Artikel nach Veröffentlichungsdatum (nach Monat gruppiert)}
    \label{article-frequency}
\end{figure}

Der so erstellte Korpus umfasst ingsgesammt 16836 Texte und deckt beginnend in 2006 einen Zeitraum von 14 Jahren ab.
Wie in Grafik \ref{article-frequency} zu erkennen ist, steigt beginnend in 2016 die Zahl der Artikel im Korpus deutlich an.
\todo{Wirklich?}
Dies liegt zum einen daran, dass einzelne der betrachteten Angebote die Veröffentlichungsfrequenz erhöhrt haben, ist aber überwiegend darauf zurückzuführen, dass nicht alle Internetangebote über den gesammten Zeitraum existiert haben sondern erst später aktiv wurden.
Genauere Informationen zu den von den einzelnen Angeboten abgedeckten Zeiträumen sowie zu Artikelzahl und Länge finden sich in Tabelle \ref{corpus-stats}.

Wie dort auch ersichtlich ist, unterscheidet sich nicht nur die absolute Zahl der Artikel nach Quelle stark, auch das Veröffentlichungsinterval schwankt von weniger als einem Artikel den Monat bis über 100.
Es besteht weiterhin ein negativer Zusammenhang zwischen der Artikellänge und dem Veröffentlichungsintervall (\todo{r richtiges Zeichen?}$r = -0.72, p = 0.065$).

\begin{table}
    \begin{center}
        \begin{tabularx}{\textwidth}{lXXXXX}
            \toprule
            & Von & Bis & Mittlere Zeichenzahl & Anzahl Artikel & Artikel/Monat\\
            \midrule
            Alles Schall und Rauch & 23.08.2006 & 02.08.2020 & 5000 & 5349 & 32.0\\
            conrebbi & 05.09.2012 & 20.10.2014 & 5610 & 24 & 1.0\\
            deutschlandpranger & 29.10.2016 & 30.12.2020 & 7527 & 118 & 2.4\\
            fm-tv & 31.07.2008 & 02.11.2018 & 9158 & 97 & 0.8\\
            hinterderfichte & 16.01.2010 & 31.05.2018 & 4221 & 1083 & 10.8\\
            \addlinespace
            recentr & 03.08.2007 & 05.08.2020 & 5071 & 4762 & 30.5\\
            Watergate.tv & 20.05.2016 & 16.11.2020 & 2810 & 5403 & 101.9\\
            \bottomrule
            \end{tabularx}
        \caption{Kennzahlen der einzelnen Quellen des Korpuses}
        \label{corpus-stats}
    \end{center}
\end{table}

Auch eine inhaltliche Betrachtung zeigt deutliche Unterschiede zwischen den einzelnen Angeboten.
So ist etwa \textit{Alles Schall und Rauch} (\textit{ASuR}) als ältestes vertretenes Angebot auch einer der "klassischsten" \todo{Das muss anders} Vertreter des Genres.
Hier wird der Nutzer direkt auf der Startseite mit Fragen konfrontiert wie "Was geschau wirklich am 11. September?" oder "Was passiert tatsächlich mit dem Klima?" \parencite{asur-homepage}. \todo{Eventuell zitation zu häufigen fragen.}

Die Seite bedient sich klassischer Verschwörungstheorien und wird im allgemeinen der Truther Szene zugeordnet \parencite{psiram-asur}.
Sie hat laut eigenaussage im Schnitt über 50.000 Zugriffe täglich\footnote{Der Besucherzähler der Website ist inzwischen nicht mehr funktional, Angabe übernommen aus \parencite{vice-asur}}.

Inhaltlich finden sich sowohl Essay-artige Artikel zu Themen wie 9/11, Bilderberger oder Klimawandel auf der Seite, als auch kürzer gehaltene Meldungen zu aktuellen Ereignissen.
Es werden, wie bei fast allen anderen Seiten im Korpus, häufig Inhalte Dritter eingebunden sowie die Artikel durch Bilder oder Grafiken ergänzt.

Ähnlich gelagert sind die Angebote \textit{Hinter der Fichte} (\textit{HdF}) und \textit{conrebbi}, beide können der Truther Szene zugerechnet werden (siehe etwa \cite{psiram-conrebbi}).
In beiden Angeboten werden aktuelle Ereignisse kommentiert und verschwörungstheoretisch interpretiert, sowie längere Essays zu typischen Themen verfasst.
Im Vergleich zu \textit{ASuR} werden bei beiden Angeboten deutlich mehr Inhalte in Form von Videos und Fremdquellen in Bild und Textform genutzt.

Etwas aus der Reihe fallen die Angebote \textit{fm-tv} sowie \textit{Deutschlandpranger}.
Inhaltlich finden sich etwa beim \textit{Deutschlandpranger} recht klassische Themen wie die Leugnung des Klimawandels \parencite*[vgl.][]{dprang-klima}, sind die dazugehörigen Text überdurchschnittlich lang und voller Werbung für Bücher zu z.T. komplett unabhängigen Themen.
Auch sind die Texte selbst teilweise relativ wirr, so enthalten beispielsweise die (sehr ausführlichen) AGB einen Abschnitt zu Strafzahlungen bei "Übersenden eines Statements anstatt einer echten Rechnung (True Bill) des wahren Haftungsgläubigers" \parencite*{dprang-agb}.

Die verbleibenden beiden Angebote \textit{recentr} und \textit{Watergate.tv} sind beide der rechten Szene zuzuordnen und verbreiten Verschwörungstheorien in diese Richtung.
Beide zeichnet ein eher journalistischer Stil aus, es werden kurze bis mittellange Meldungen veröffentlicht, häufig begleitet von eigenen oder fremden Videos.
\textit{Recentr} wurde 2006 als deutscher Ableger von \textit{Inforwars.com} gegründet und bedient sich noch heute einem ähnlichen Konzept.
Ähnlich dem Vorbild wird massiv Werbung für den eigenen Shop gemacht in dem diverse Produkte zur Vorbereitung auf apokalyptische Szenarien sowie fragwürdige Nahrungsergänzungsmittel verkauft werden.
Inhaltlich ist vor allem interessant, dass sich die Seite (bzw. der Hauptautor Alexander Benesch) z.T. sehr explizit von anderen Teilen der rechten/verschwörungstheoretischen Szene abgrenzt.
\footnote{So grenzt sich das Portal in Beiträgen inzwischen auch von \textit{Inforwars} Gründer Alex Jones ab \parencite*{recentr-jones} oder spekuliert darüber, ob die Alternative für Deutschland vom britischen Geheimdienst gegründet wurde \parencite{recentr-afd}}
So kommt es durchaus vor, dass in einem Artikel zunächst von der "implausiblen Sichtweise, verschiedenste Regierungen hätten eine Fake-Pandemie orchestriert" \parencite{recentr-population} gesprochen wird, nur um im nächsten Satz zu erklären: 

\begin{quotation}
    In Wirklichkeit sind die Methoden der globalen Bevölkerungsreduktion fast ausschließlich legal, simpel und werden schrittweise angewandt [...]. Das Konzept ist so entworfen, um mit möglichst wenig Zwang auszukommen [...]. \parencite{recentr-population}
\end{quotation} 

Die Website \textit{Watergate.tv} (inzwischen Teil von \textit{NEOPresse.com}) ist ein rechtslastiges Nachrichten und Videoportal, das sich verschwörungstheoretischer Narrative bedient. 
Auf dem Portal finden sich klassische Themen wie Bilderberger \parencite[vgl.][]{watergate-bilderberger} aber auch Fake-News und Berichterstattung mit starker pro-russischen Tendenz.
In die öffentlichkeit Rückte das Portal als sich das Neo-Magazin-Royale wegen seiner Mitarbeit bei \textit{Watergate.tv} von Hans Meiser trennte \parencite*[Siehe z.B.][]{spiegel-watergate}.

\todo{Kurzes Fazit Korpus}

% Vergleichskorpus

Bei der Erstellung des Vergleichskorpus war es das Ziel Quellen zu wählen die dem Korpus konspirativer Texte möglichst ähnlich sind, ohne die konspirativen Aspekte zu enthalten.
Da insbesondere die letzten beiden besprochenen Angebote sich inhaltlich sehr an journalisitsche Angebote anlehnen, lag es nah ebensolche heranzuziehen.
Aus Gründen der Vollständigkeit (Die Angebote sollten möglichst den kompletten Zeitraum des Korpuses abdecken) sowie der Zugänglichkeit wurden die Angebote von Spiegel Online sowie der Frankfurter Rundschau ausgewählt.
Es wurde für beide Onlineangebote zunächst ein Index aller Artikel die in den selben Zeitraum wie der Korpus fallen erstellt und aus diesem zufällig eine Auswahl von 10000 Artikeln pro Anbieter gezogen.

Die restlichen im Korpus enthaltenen Seiten sind alle mehr oder weniger in Blogform gehalten.
Um Inhalte in einer vergleichbaren Form zu findne aber auch dem in der Literatur häufig anzutreffenden Feststellungen der wissenschaftlichen Form von verschwörungstheoretischen Texten Rechnung zu tragen wurde eine Reihe von Wissenschaftsblogs als zweite Komponente im Vergleichskorpus gewählt.
Dafür wurden von den beiden Platformen scienceblogs und scilogs\footnote{Beide Platformen sind Anbieter bei denen eine Vielzahl verschiedener Blogs zu finden sind.} ähnlich den journalistischen Angeboten aus dem gesamten Index der Einträge eine Auswahl von jeweils 10000 Artikeln ausgewählt.
Genauere Kennwerte zum Vergleichskorpus siehe in Tabelle \ref{comcorpus-stats}.

\begin{table}
    \begin{center}
        \begin{tabularx}{\textwidth}{XXXXXX}
            \toprule
            & Von & Bis & Mittlere Zeichenzahl & Anzahl Artikel & Artikel/Monat\\
            \midrule
            Frankfurter Rundschau & 03.08.2006 & 04.08.2020 & 3670 & 9959 & 59.3\\
            scienceblogs & 14.02.2007 & 04.08.2020 & 3261 & 8765 & 54.4\\
            scilogs & 29.12.2000 & 03.08.2020 & 5206 & 9533 & 40.6\\
            Spiegel Online & 01.08.2006 & 07.01.2020 & 4687 & 8665 & 53.8\\
            \bottomrule
        \end{tabularx}
        \caption{Kennzahlen der einzelnen Quellen des Vergleiskorpuses}
        \label{comcorpus-stats}
    \end{center}
\end{table}

\subsection{Datenvorverarbeitung}

Zunächst wurden einige Säuberungsschritte auf den Korpus angewendet, um zu verhindern, dass fehlerhafte oder irrelevante Daten mit für die Feature Erstellung herangezogen werden.

Es wurde zunächst Unicode Normalisierung für alle Texte durchgeführt, anschließend wurden mittels des \textit{Google Compact Language Detector 3} \parencite[][]{cld3} die Sprache aller Text bestimmt nur (überwiegend) deutschsprachige Texte zu betrachten.
Da im Korpus an einigen Stellen noch nicht zum Artikeltext gehörige Komponenten aus vielen Sonderzeichen vorhanden waren, wurden alle Wörter entfernt, die nur aus Zeichen bestanden, die unter 1000 Mal im gesammten Korpus vorkamen.
Schlussendlich wurden noch kleinere Korrekturen vorgenommen, wie das Zusammenfassen multipler, aufeinanderfolgender Leerzeichen und Unterstriche (die gerne als Trennlinien in den Text benutzt wurden).

Schlussendlich wurden noch alle Text unter 100 Zeichen länge aus dem Korpus entfernt, da diese für die Klassifizierungsaufgabe nur wenig Informationen enthalten können. Der finale Korpus umfasste 53758 Texte, mit einem Anteil der positiven Klasse (sprich von verschwörungstheoretischen Texten) von etwa 32\%.

\subsection{Feature Erstellung}



% Ergebnisse

\section{Diskussion}

Das kleinere, ausschließlich mit literaturbasierten Features trainierte Modell erreichte mit 87\% bereits eine hohe Präzision.
Dies kann dahingehend interpretiert werden, dass die genutzten Features grundsätzlich geeignet sind um verschwörungstheoretische Texte von (wissenschafts-)journalistischen zu Unterscheiden.

Die verwendete \textit{LightGBM} Software erlaubt es die Bedeutung einzelner Features für das erstellte Modell abzuschätzen.
Das Feature mit dem höchsten Informationszuwachs (Gain) für das Modell ist die Anzahl der internen Links (Gain = $0.12$).
Die Anzahl der externen Links hingegen bot einen deutlich geringeren Gain ($0.076$) für das Modell.
Dies ist insofern bemerkenswert, als dass die in der Literatur gemachten Beobachtungen wie etwa die von \textcite[10]{soukup_2008} hätten vermuten lassen, dass die Verknüpfung mit externen Quellen ein wichtigeres Merkmal ist als die mit internen.

Die nächstwichtigsten Features sind die Anzahl der eingebundenen Bilder (Gain = $0.1$), die Summe der Sentimente ($0.1$) und der Anteil an zitierten Text ($0.9$).
Interessant ist hier noch, dass, anders als die Summe an Sentiment Scores, der Betrag der Sentiment Scores für die Klassifizierung relativ unbedeutend ist (Gain = $0.018$).
Dies kann zum einen daran liegen, dass diese Werte nicht vollständig unabhängig voneinander sind und der Trainingsalgorithmus deshalb u.U. nur einen der Werte berücksichtigt.\footnote{Baum-basierte Verfahren sind meist in der Lage relevante Features zu einem gewissen Grad selbst zu selektieren.}
Zum anderen kann es aber auch ein Indiz dafür sein, dass Verschwörungstheoretische Texte nicht notwendigerweise allgemein emotionaler sind, aber im Schnitt deutlich negativere Sentimente Transportieren.
Dies bestätigt sich auch an den Durchschnittswerten der Sentimentsumme ($\overline{X}_1 = -4.22$ gegenüber $\overline{X}_0 = -1.35$).

Die meisten anderen Features lieferten dem Modell einen mittelgroßen Informationszuwachs, besonders wenig nützlich waren vor allem Features, die mehr oder weniger mit anderen Features zusammenhängen (die Summe positiver bzw. negativer Sentimente, Zahl der direkten Zitate, etc.), was durchaus erwartbar ist.
Ebenso relativ wenig hilfreich waren die Menge an Negationen, an Zahlenangaben sowie allgemein an eingebetteten Medien (Twitter, Youtube, Sonstige).
Eine Erklärung für letztere könnte dabei aber auch schlicht sein, dass herkömmliche Medien inzwischen solche Medien auch häufiger einbinden und sich in diesem stilistischen Merkmal den Verschwörungstheorien angenähert haben.

Es bleibt anzumerken, dass Verfahren des maschinellen Lernens, wie hier Gradienten-Boosting, komplexe Systeme sind und die Interpretation solcher Ergebnisse immer mit Vorsicht vorgenommen werden sollte.

\begin{figure}[h]
    \centering
    \includegraphics[scale=0.45]{graphics/top_10_features.jpg}
    \caption{Top-10 Features nach Bedeutung für das Modell. (Namen in Großbuchstaben = Part-of-Speech Tag; Namen in Kleinbuchstaben = tf-idf Worthäufigkeit; Benannte Features = Auf Literatur basierende Statistiken)}
    \label{top-features}
\end{figure}

\FloatBarrier

Das vollständige Modell erreichte mit fast 98\% eine sehr gute Präzision, auch Spezifität und Sensitivität sind auf ähnlich hohem Niveau.
Es kann zuverlässig zwischen den Verschwörungstheoretischen Texten und dem Vergleichskorpus unterscheiden.
Eine Betrachtung der wichtigsten Features aus Grafik \ref{top-features} zeigt zunächst, dass viele der auch für das kleinere Model wichtigen Features wieder von großer Bedeutung sind (Zahl der Bilder, Zahl der internen Links, Summe der Sentiment Scores und Anteil von zitiertem Text).
Einige der Wortfrequenzen mit dem meisten Einfluss auf die Klassifizierungen sind Foto, Krieg, Merkel und Regierung.
Die meisten davon sind wenig überraschende Ergebnisse, die Verwendung von Worten aus Themenbereichen wie Krieg und Macht ist in der Literatur gut belegt, etwa bei \textcite[150]{stumpf_2019} oder \textcite[25]{uscinski_2014}.
Unter den wichtigen Part-of-Speech Tags sind Leerzeichen, Adverben, fremdsprachiger Text und Coordinating Conjunctions (also Wörtern wie und, oder, aber, etc. \parencite[vgl.][]{smith_2003}).
Interessant ist auch, dass die Zahl der Fragezeichen im kleineren Modell zwar kaum relevant ist, im größeren Modell aber unter den 10 einflussreichsten Features.

Eine mögliche Erklärung für die überraschend gute Leistung des trainierten Modells ist, dass die relativ geringe Vielfalt, insbesondere im Untersuchungskorpus, einen Teil zu der guten Leistung beigetragen hat.
Auch wenn etwa für \textit{Watergate.tv} mehrere Autor:innen schreiben, ist zu vermuten dass die totale Anzahl an Autor:innen im Korpus im niedrigen zweistelligen Bereich liegt.
Da in der Modellerstellung mit den Part-of-Speech Tags auch eher stilistische Features genutzt werden, ist es durchaus möglich, dass ein Teil der Leistung des Modells weniger auf verschwörungstheoretischen Merkmalen beruht als auf dem Stil der individuellen Autor:innen.

Dies könnte ein Faktor sein, der es unwahrscheinlicher macht, dass das vorgestellte Modell gut generalisiert.

% Fazit

\section{Fazit}

% Rückgriff auf die Vorgestellten automatischen Ansätze

\clearpage

%
% ---- Bibliography ----
%
%
\printbibliography[title=Literatur, keyword=literature]
\printbibliography[title=Software, keyword=software]
\printbibliography[title=Online, keyword=online]

\end{document}

\section{Fazit}

Es wurde basierend auf einem Korpus aus Texten deutschsprachiger Verschwörungstheoretischer Texte und einem Vergleichskorpus aus (wissensschafts-)journalistischen Texten ein Modell trainiert, dass zuverlässig die verschwörungstheoretischen Texte identifizieren kann.
In der Literatur gemachte Beobachtungen zur literarischen Form von Verschwörungstheorien wurden dabei aufgegriffen und als Features in das Modell eingebracht.
Diese leisteten einen wichtigen Beitrag für die Funktion des trainierten Modells, es konnte so gezeigt werden, dass einige dieser qualitativen Beobachtungen auch für eine qualitative Analyse von verschwörungstheorien geeignet sind.
Es wurden weiterhin Wortfrequenzen und Part-of-Speech Tags als zusätzliche Features in das Modell eingebracht.
Das so entstandene Modell arbeitet deutlich genauer.

Der vorgestellte Ansatz konnte eine gute Leistung erbringen, ist dabei aber deutlich weniger komplex als die bisherigen Verfahren zur automatischen Verarbeitung von Verschwörungstheorien.
Er ist dafür allerdings u.U. deutlich unschärfer, es wird keine spezifische Verschwörungstheoretische Aussage identifiziert wie dies etwa \textcite[]{samory_2018} tut, vielmehr wird ein allgemeiner verschwörungstheoretischer Stil identifiziert.

Ein mögliches Problem des vorgestellten Verfahrens könnte auch die Generalisierbarkeit des trainierten Modells sein.
Ein möglicher Ansatzpunkt um hier Verbesserungen herbeizuführen wäre die Zusammenstellung eines bedeutend größeren und diverseren Korpuses, sowie auch eines entsprechend Vergleichskorpuses vonnöten.
Auch bedarf es deutlich mehr an quantitativer Grundlagenarbeit zu sprachlichen Eigenschaften von Verschwörungstheorien um in der Zukunft bessere Systeme zu deren Erkennung zu entwickeln.


\section{Einleitung}

Verschwörungstheorien sind aktuell häufig im Zentrum der Aufmerksamkeit.
Während dieser, aktuelle Aufmerksamkeitsschub inhaltlich auf die aktuelle Pandemie zurückzuführen ist, bietet das Internet und die Verbreitung sozialer Medien das technische Fundament für die schnelle Verbrietung wie sie aktuell stattfindet.
Das zunahmende Vernetzung solchen Theorien vorschub leistet ist keine neue Entwicklung, sondern wird schon länger im wissenschaftlichen Diskurs über Verschwörungstheorien beobachtet.
Die wissenschaftliche Auseinandersetzung mit Verschwörungstheorien ist trotz des verstärkten Interesses in der jüngsten Vergangenheit bis heute eher verhalten.

Die komplexe Natur von Verschwörungstheorien stellt die wissenschaftliche Auseinandersetzung mit ihnen vor einige Grundlegende Probleme.
Das offensichtlichste davon ist sicherlich die Frage was überhaupt eine Verschwörungstheorie ist.
Hierzu existiert vor allem in der Wissenssoziologie eine Vielzahl von Literatur, ein klarer Konsens über eine Definition von Verschwörungstheorien fehlt jedoch bislang.
Insbesondere für die Arbeit mit Beiträgen aus sozialen Netzen und anderen Online-Quellen schließt sich die Frage an, wie Verschwörungstheoretische Beiträge zuverlässig identifiziert werden können.
Der Nutzen solcher Techniken nicht nur innerhalb der Wissenschaft ist naheliegend, so versuchen auch die Betreiber von sozialen Netzwerken daran verschwörungstheoretische Inhalte automatisiert zu erkennen (Beispiel google/youztube).
In der Forschung wird dieses Problem klassischerweise mit der Hilfe einer Vielzahl von Hilfskräften gelöst, die die Daten manuell bearbeiten. (Studie mit codierern)

Diese Arbeit stellt einen Ansatz vor, dieses Problem mit Technologien der automatischen Sprachverarbeitung und des maschinellen Lernens zu lösen.
Primär werden dabei stilistische Merkmale der untersuchten Texte genutzt und nur sekundär inhaltliche Informationen wie das Verwenden von Schlüsselwörtern etc.
Durch das Verwenden solcher Technologien kann zum einen der Resourcenaufwand drastisch reduziert werden, aber auch die Abhängigkeit von komplexen und damit fehleranfälligen Kriterien für die Klassifizierung lässt sich so umgehen.
Auch ist der hier vorgestellte Ansatz durch den Fokus auf stilistisceh Merkmale verhältnissmäßig robust gegenüber neuen inhaltlichen Entwicklungen der Verschwörungstheorien und somit potentiell zukunftssicherer als vergleichbare Ansätze.

Im folgenden soll zunächst ein kurzer Abriss über den aktuellen Stand der wissenschaftlichen Auseinandersetzung mit Verschöwrungstheorien allgemein, sowie der Forschung zu linguistischen Merkmalen im speziellen, gegeben werden.
Anschließend soll kurz auf die verwendeten Technologien sowie die Datengrundlage dieser Arbeit gegeben werden, bevor konkrete Ergebnisse diskutiert werden.
\section{Einleitung}

Verschwörungstheorien sind aktuell stark im Zentrum der öffentlichen Aufmerksamkeit.
Während dieser Aufmerksamkeitsschub vor allem auf die Covid-19 Pandemie zurückzuführen ist, erleben Verschwörungstheorien schon seit Anfang der 2000er Jahre einen kontinuierlichen Popularitätsanstieg.
Dieser ist vor allem begründet in der zunehmenden Verbreitung des Internets und der sozialen Medien \parencite[vgl.][492]{stano_2020}.

Eine der Kernfragen in der wissenschaftlichen Auseinandersetzung mit Verschwörungstheorien ist die Frage was eine Verschwörungstheorie ist und was nicht.
Eine theoretische Auseinandersetzung mit dieser Frage findet insbesondere in der Wissensoziologie und der Philosophie statt.\footnote{Siehe z.B. \textcite[]{coady_2006} oder auch \textcite[23-53]{uscinski_2014}.}

Weitgehend unbeantwortet ist hingegen, wie sich diese Frage quantitativ beantworten lässt.
Die wenigen hierzu veröffentlichten Arbeiten nutzen meist verhältnismäßig spezifische und komplexe Methoden.\footnote{Siehe etwa \textcite[]{samory_2018} oder \textcite[]{shahsavari_2020}.}

Diese Arbeit stellt einen Ansatz vor, der versucht dieses Problem mit Methoden des Text-Minings und des maschinellen Lernens zu lösen.
Der vorgestellte Ansatz ist dabei deutlich allgemeiner als die in der Literatur vertretenen und identifiziert verschwörungstheoretische Artikel mittels sprachlicher und stilistischer Merkmale.

Es wird dafür ein Korpus aus Artikeln von sieben Webportalen der deutschsprachigen, verschwörungstheoretischen Szene ausgewertet. 
Aus diesem werden Features extrahiert, die anschließend genutzt werden, um ein Modell zu trainieren, welches diese Artikel von (wissenschafts-)journalistischen Artikeln in einem Vergleichskorpus unterscheiden kann.
Dabei erfolgt ein Rückgriff auf in der bestehenden, überwiegend qualitativen Forschung gemachten Erkenntnisse und Beobachtungen.
Ebenfalls werden Methoden aus der natürlichen Sprachverarbeitung verwendet, um Features aus Wortfrequenzen und Part-of-Speech Tags zu erstellen.

Das erstellte Modell kann präzise verschwörungstheoretische Artikel identifizieren und von denen im Vergleichskorpus unterscheiden.
Die dabei gemachten Beobachtungen können dabei helfen in der qualitativen Forschung gemachte Beobachtungen quantitativ zu validieren.
Auch könnte die hier vorgestellte Methode hilfreich sein um verschwörungstheoretische Inhalte in größeren Datenmengen zu identifizieren.
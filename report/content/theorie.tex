\section{Theorie}

% Forschungsstand, Methoden zur Identifizierung

\subsection{Stand der Forschung}

Der Beginn der Wissenschaftlichen auseinandersetzung mit Verschwörungstheorien wird üblicherweise rund um das Erscheinen von Richard \textcite{hofstadter_2008} Essay \textit{The Paranoid Style in American Politics} verortet.
In diesem setzt Hofstadter sich mit dem was er für von Verschwörungstheorien beeinflusste Politik in den USA hält, wie etwa die Schriften von Senator McCarthy und die nach ihm benannte Praxis des McCarthyismus \todo{Maybe citation needed}.

Das weitere Forschungsinteresse im restlichen 20. Jahrhundert hielt sich in Grenzen, die Verbleibende Forschung fiel meist in eine von Zwei Kategorien, wie \textcite{sunstein_2008} passend zusammenfassen:

\begin{quotation}
    The academic literature on conspiracy theories is thin, and most of it falls into one 
    of two classes: (1) work by analytic philosophers, especially in epistemology and the 
    philosophy of science, that asks what counts as a “conspiracy theory” and whether such theories are methodologically suspect; (2) a smattering of work in sociology and Freudian psychology on the causes of conspiracy theorizing. \parencite[][2]{sunstein_2008}
\end{quotation}

In der jüngeren Vergangenheit haben vor allem Drei Entwicklungen Verschwörungstheorien oder zumindest deren Sichtbarkeit und damit auch der Forschung in diesem Bereich vorschub geleistet.

Dies wäre zum einen die Verbreitung des Internets um die Jahrtausendwende.
Die Mehrheit der Autoren sieht hier darin ein erhebliches Hilfsmittel für die Verbreitung von Verschwörungstheorien, wie etwa \textcite{stano_2020} schreibt: "[...] The Internet, and in particular social networks, have proved fundamental to the spread and development of such [conspiracy] theories" \parencite[][492]{stano_2020}.\footnote{Es gilt jedoch anzumerken, dass auch die gegenteile Auffassung in der Literatur vertreten ist, so etwa bei \textcite{clarke_2007}.}

In der jüngsten Vergangenheit haben besonders die Wahl Donald Trumps zum US-Präsidenten 2016 und die aktuelle COVID-19 Pandemie Verschwörungstheorien ins öffentliche Interesse gerückt.
Wärend die wissenschaftliche Aufarbeitung insbesondere von letzterer noch am Anfang steht, sind hier bereits erste Arbeiten vorhanden die neue Ansätze zeigen, wie etwa die von \textcite{shahsavari_2020}.

Eines der Kernprobleme der Forschungs zu Verschwörungstheorien, ist die Frage wie Verschwörungstheorien als solche zu identifizieren sind.
Diese Frage hat nicht nur eine fundamentale Rolle für theoretische Arbeiten zu Verschwörungstheorien, sondern ist auch von immenser praktischer Relevanz in einer Zeit in der in sozialen Medien die Grenzen zwischen Nachrichten, Verschwörungstheorien und Fake-News immer schwerer auszumachen sind.\todo{Eventuell google algorithmus zitieren}

In der Vergangenheit war die wissenschaftliche Beantwortung dieser Frage meist eng mit der Frage danach verbunden was eine Verschwörungstheorie ausmacht und anhand welcher kriterien sich dies bestimmen lässt.
Eine Beispielhafte Arbeit ist hier \textcite{uscinski_2014}.
Die Autoren nutzen darin 6 verschiedene Tests um Verschwörungstheorien zu identifizieren, als Beispiel sei Occamm`s Razor, also die Frage nach der einfachsten Erklärung, und die falsifizierbarkeit einer Theorie genannt.
Die Autoren selbst werten mit diesen Krieterien und der Methode der Inhaltsanalyse sowie einer Vielzahl von studentischen Kodierer:Innen einen Korpus an Briefen an US-Redaktionen aus \parencite[54ff]{uscinski_2014}.
Während die Autoren überzeugend dafür argumentieren, alle der aufgestellten Kriterien für eine Klassifizerung einzusetzten \parencite[52f]{uscinski_2014}, ist keines der eingeführten Kriterien, zumindest mit den aktuellen technischen möglichkeiten, wirklich geeignet um automatisierte Klassifikationen vorzunehmen.

Demgegenüber sind in jüngster Vergangenheit erste Arbeiten entstanden, die sich mit automatisieren Verfahren zum Erkennen von Verschwörungstheorien beschäftigen.
Hier ist vor allem die Arbeit von \textcite{samory_2018} hervorzuheben, in der die Autor:Innen automatisiert Triplets aus \textit{agent}, \textit{action} und \textit{target} aus Verschwörungstheoretischen Online Beiträgen extrahieren.
Als Datengrundlage nutzen sie dazu Beiträge aus dem Reddit Subforum r/conspiracy.
Diese Werten sie mittels einer NLP Pipeline aus die unter anderem Topic Modeling, Dependency Parser und Wordvektoren nutzt, aber auch vereinzelt Expertenwissen einbringt \parencite[][6ff]{samory_2018}.
Der Ansatz stellt einen wichtigen Beitrag zu einer vollständig automatisierten Analyse dar, ist aber auch verhältnismäßig komplex und basiert auf einer Datenbasis die nicht notwendigerweise Repräsentativ für "wild" vertretene Verschwörungstheorien ist, so dass die Übertragbarkeit für einen Identifizierungsmechanismus fragwürdig scheint.

Eine weitere Arbeit die sich mit automatisierten Auswertungen beschäftigt, ist die von \textcite{shahsavari_2020}.
Die Autor:innen werten dabei Online Diskussionen in Foren wie Reddit und 4Chan aus, die sich mit der Covid 19 Pandemie beschäftigen.
Die Auswahl ob Diskussionen verschwörungstheoretische Inhalte enthalten oder nicht, wurde dabei von einem Expertengremium getroffen \parencite[284f]{shahsavari_2020}.
Es werden mittels Methoden aus der Natürlichen Sprachverarbeitung narrative Netzwerke ausgewertet und so Verbindungen zu den inhalten von Nachrichten und den aktuell "beliebtesten" Verschwörungstheorien aufgezeigt. \todo{Hier eventuell mehr/besser belegen/recherchieren}
Während auch diese Arbeit zweifelsfrei einen wichtigen Beitrag zu einem besseren Verständnis insbesondere der Verbreitung von Verschwörungstheorien in sozialen Netzwerken darstellt, wird das Problem der identifizierung von Verschwörungstheorien damit nicht wirklich gelöst.

\subsection{Merkmale von verschwörungstheoretischen Texten}

Um ebendiese Aufgabe angehen zu können, soll zunächst ein Blick in die Literatur erfolgen und in die in der qualitativen und quantitativen Forschung gemachten Beobachtungen zu sprachlichen Merkmalen von Verschwörungstheorien.
Eine erschöpfende Auflistung aller in der Literatur zu findenden sprachlichen Merkmale von Verschwörungstheorien ist insbesondere mit dem erstarkten Forschungsinteresse in der jüngsten Vergangenheit kaum möglich noch wäre es zielführend.
Es sollen daher hier vor allem solche Merkmale genannt werden, die sich für eine quantitative Erfassung eignen und in dieser Arbeit 

Eine in der Literatur häufig gemachte Beobachtung ist die emotionalität der Argumentation in Verschwörungstheorien \parencite[Vgl.][10]{miller_2002}.
Dieses Argument findet sich auch bei \textcite[][93ff]{butter_2018}, der hier noch spezifischer darauf eingeht, dass es vor allem die vermeintlichen Verschwörer sind denen mit "metaphorisch aufgeladener, bisweilen apokalyptischer Sprache ausschließlich negative Eigenschaften zugeschrieben [werden]" \parencite[][93f]{butter_2018}.

Gleichzeitig spricht der Autor aber auch davon, dass sich Verschwörungstheoretiker traditionell um eine seriöse Darstellung bemühen und sich der verwendete Stil an den der Wissenschaft anlehne \parencite[][61]{butter_2018}.
Diese Feststellung findet sich bereits bei \textcite{hofstadter_2008}:

\begin{quotation}
    The higher paranoid scholarship is nothing if not coherent—in fact the paranoid mind is far more coherent than the real world. It is nothing if not scholarly in technique. McCarthy’s 96-page pamphlet, McCarthyism, contains no less than 313 footnote references, and Mr. Welch’s incredible assault on Eisenhower, The Politician, has one hundred pages of bibliography and notes. \parencite[][37]{hofstadter_2008}
\end{quotation}

Seit der Erstveröffentlichung von Hofstadters Aufsatz hat vor allem die Existenz des Internets zu Veränderungen in der Erzählart von Verschwörungstheorien geführt.
Die obsession an Belegen und Referenzen existiert zwar weiterhin, hat sich aber insoweit Verändert, als das nicht mehr unbedingt Fussnoten das Mittel der Wahl sind, vielmehr werden Inhalte direkt eingebunden oder verlinkt.
So schreibt etwa \textcite{soukup_2008} der sich spezifisch mit Verschwörungstheorien um den 11. September im Internet befasst, diese müssten als "digital, hypertextual, and multimedial experience" \parencite[10]{soukup_2008} verstanden werden.

Erst in der jüngeren Vergangenheit finden sich Arbeiten die sich spezifisch mit den linguistischen Merkmalen von Verschwörungstheorien auseinandersetzen.
\textcite{schafer_2018} etwa analysiert einen Korpus aus Kommentaren zu verschwörungstheoretischen YouTube-Videos.
Sie stellt dabei unter anderem eine gehäufte Verwendung von Ironie und \textit{scare quotes} fest, wenn es darum geht die Gegenseite negativ darzustellen \parencite[235]{schafer_2018}.
Ähnlich der bereits erörterten Literatur findet die Autorin auch eine hohe Dichte von Belegen und Referenzen die die Kompetenzen der VerschwörungstheoretikerInnen unterstreichen sollen, stellt aber noch heraus, dass diese sich häufig durch nennen von Namen und Zahlenangaben ausdrücken \parencite[234]{schafer_2018}.

Eine ähnliche Beobachtung macht auch \textcite{filatkina_2018}, welche feststellt:

\begin{quotation}
    In einem Beitrag oft mehrfach angeführte Verweise auf offizielle Zahlen [...] bzw. das wörtliche Zitieren der VertreterInnen der Wissenschaft und Politik sollen die Glaubwürdigkeit herstellen. \parencite[][208]{filatkina_2018}
\end{quotation}

Ebenfalls stellt die Autorin fest, dass eine übliche Konstruktion darin besteht, durch offene und geschlossene Fragen die Aufmerksamkeit der LeserInnen auf bestimmte Aspekte zu lenken \parencite[][205]{filatkina_2018}.

Schlussendlich findet sich bei \textcite[149]{stumpf_2019} noch die Feststellung, dass häufig Negationswörter verwendet werden um die offiziellen Aussagen zu verneinen.


% Vokabular:
%  - Entlarfungsvokabular \parencite[][51]{ebling_2013}
%  - Stumpf/Römer 2019 haben viel Vokabular

\subsection{Textklassifizierung}

% Hier über klassische Ansätze für Textklassifizierung etc.
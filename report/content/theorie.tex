\section{Theorie}

% Allgemein Verschwörungstheorien, Stand der Forschung etc

\subsection{Merkmale von verschwörungstheoretischen Texten}

Eine erschöpfende Auflistung aller in der Literatur zu findenden sprachlichen Merkmale von Verschwörungstheorien ist insbesondere mit dem erstarkten Forschungsinteresse in der jüngsten Vergangenheit kaum möglich noch wäre es zielführend.
Es sollen daher hier vor allem solche Merkmale genannt werden, die sich für eine quantitative Erfassung eignen und in dieser Arbeit 

Eine in der Literatur häufig gemachte Beobachtung ist die emotionalität der Argumentation in Verschwörungstheorien \parencite[Vgl.][10]{miller_2002}.
Dieses Argument findet sich auch bei \textcite[][93ff]{butter_2018}, der hier noch spezifischer darauf eingeht, dass es vor allem die vermeintlichen Verschwörer sind denen mit "metaphorisch aufgeladener, bisweilen apokalyptischer Sprache ausschließlich negative Eigenschaften zugeschrieben [werden]" \parencite[][93f]{butter_2018}.

Gleichzeitig spricht der Autor aber auch davon, dass sich Verschwörungstheoretiker traditionell um eine seriöse Darstellung bemühen und sich der verwendete Stil an den der Wissenschaft anlehne \parencite[][61]{butter_2018}.
Diese Feststellung findet sich bereits bei \textcite{hofstadter_2008}:

\begin{quotation}
    The higher paranoid scholarship is nothing if not coherent—in fact the paranoid mind is far more coherent than the real world. It is nothing if not scholarly in technique. McCarthy’s 96-page pamphlet, McCarthyism, contains no less than 313 footnote references, and Mr. Welch’s incredible assault on Eisenhower, The Politician, has one hundred pages of bibliography and notes. \parencite[][37]{hofstadter_2008}
\end{quotation}

Seit der Erstveröffentlichung von Hofstadters Aufsatz hat vor allem die Existenz des Internets zu Veränderungen in der Erzählart von Verschwörungstheorien geführt.
Die obsession an Belegen und Referenzen existiert zwar weiterhin, hat sich aber insoweit Verändert, als das nicht mehr unbedingt Fussnoten das Mittel der Wahl sind, vielmehr werden Inhalte direkt eingebunden oder verlinkt.
So schreibt etwa \textcite{soukup_2008} der sich spezifisch mit Verschwörungstheorien um den 11. September im Internet befasst, diese müssten als "digital, hypertextual, and multimedial experience" \parencite[10]{soukup_2008} verstanden werden.

Erst in der jüngeren Vergangenheit finden sich Arbeiten die sich spezifisch mit den linguistischen Merkmalen von Verschwörungstheorien auseinandersetzen.
\textcite{schafer_2018} etwa analysiert einen Korpus aus Kommentaren zu verschwörungstheoretischen YouTube-Videos.
Sie stellt dabei unter anderem eine gehäufte Verwendung von Ironie und \textit{scare quotes} fest, wenn es darum geht die Gegenseite negativ darzustellen \parencite[235]{schafer_2018}.
Ähnlich der bereits erörterten Literatur findet die Autorin auch eine hohe Dichte von Belegen und Referenzen die die Kompetenzen der VerschwörungstheoretikerInnen unterstreichen sollen, stellt aber noch heraus, dass diese sich häufig durch nennen von Namen und Zahlenangaben ausdrücken \parencite[234]{schafer_2018}.

Eine ähnliche Beobachtung macht auch \textcite{filatkina_2018}, welche feststellt:

\begin{quotation}
    In einem Beitrag oft mehrfach angeführte Verweise auf offizielle Zahlen [...] bzw. das wörtliche Zitieren der VertreterInnen der Wissenschaft und Politik sollen die Glaubwürdigkeit herstellen. \parencite[][208]{filatkina_2018}
\end{quotation}

Ebenfalls stellt die Autorin fest, dass eine übliche Konstruktion darin besteht, durch offene und geschlossene Fragen die Aufmerksamkeit der LeserInnen auf bestimmte Aspekte zu lenken \parencite[][205]{filatkina_2018}.

Schlussendlich findet sich bei \textcite[149]{stumpf_2019} noch die Feststellung, dass häufig Negationswörter verwendet werden um die offiziellen Aussagen zu verneinen.


% Vokabular:
%  - Entlarfungsvokabular \parencite[][51]{ebling_2013}
%  - Stumpf/Römer 2019 haben viel Vokabular